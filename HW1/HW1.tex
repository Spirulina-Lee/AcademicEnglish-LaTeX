\documentclass[a4paper,journal]{IEEEtran}
\usepackage{amsmath,amsfonts}
\usepackage{algorithmic}
\usepackage{algorithm}
\usepackage{array}
\usepackage[caption=false,font=normalsize,labelfont=sf,textfont=sf]{subfig}
\usepackage{textcomp}
\usepackage{stfloats}
\usepackage{url}
\usepackage{verbatim}
\usepackage{graphicx}
\usepackage{cite}
\hyphenation{op-tical net-works semi-conduc-tor IEEE-Xplore}
% updated with editorial comments 8/9/2021

\begin{document}

\title{One of My Interests}

\author{Li Xianzhe
        % <-this % stops a space
\thanks{Manuscript revised September 24, 2024.}}

% The paper headers
\markboth{Academic English, September~2024}%
{Shell \MakeLowercase{\textit{et al.}}: One of My Interests}

\maketitle

\section{Introduction}
\IEEEPARstart{I}{ have} a strong interest in the application of deep neural networks (DNNs) in image recognition, mainly in the following aspects: technical importance, practical applications, and the potential to advance AI research. DNNs are inspired by the human brain and are made up of multiple layers of artificial neurons that work together to process information. Therefore, in image recognition tasks, DNNs can automatically detect special features such as shapes and textures directly from the raw data without the need for manual extraction. 
This ability to learn on their own has enabled the DNN to reach a high level in tasks that previously required human labor. My interest in this topic stems from its technology, real-world applications, and its research boundaries.


\section{Technical Importance}
DNNs are composed of multiple layers that work together to process information and extract hierarchical features. In image recognition, this allows DNNs to automatically detect special features such as shapes and textures directly from raw data without the need for manual feature extraction. This capability of self-learning enables DNNs to perform at levels previously requiring human intelligence.

\section{Practical Applications}
DNNs are widely used in practical applications and have had a significant impact. For example, in healthcare, DNNs help analyze medical images and sometimes identify lesions earlier than traditional methods. Assistive driving vehicles rely on DNNs to recognize objects such as pedestrians and traffic signs to ensure safe driving. In the field of security, DNN-based facial recognition systems allow us to enter campuses and dormitories more conveniently. These real-world applications demonstrate how DNNs can solve real problems and have a profound impact on society.

\section{Potential to Advance AI Research}
Despite their success, DNNs still face challenges, such as the large datasets and computational resources required during the training process for deep networks. I am interested in how researchers can improve the efficiency of these models, enhance transfer learning capabilities, and improve the interpretability of networks.

\section{Conclusion}
In summary, my interest in deep neural networks for image recognition is driven by the field’s technical sophistication, its broad applications, and the ongoing research challenges. DNNs not only reflect the current state of AI development but also point toward the future, where machines will continue to enhance their ability to "see" and understand the world around them.



\end{document}
